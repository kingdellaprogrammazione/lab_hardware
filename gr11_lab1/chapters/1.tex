\begin{figure}[h]

    \centering
\begin{quantikz}
\lstick{$\ket{0}$}   & \gate{H} \slice{$\ket{\psi_{t1}}$} & \ctrl{1} \slice{$\ket{\psi_{t2}}$} & \gate{H} \slice{$\ket{\psi_{t3}}$} & \gate{X}  \slice{$\ket{\psi_{t4}}$} & \ctrl{1} \slice{$\ket{\psi_{t5}}$} & \gate{X} \slice{$\ket{\psi_{t6}}$} & \gate{H} \slice{$\ket{\psi_{t7}}$} &  \meter{} \\
\lstick{$\ket{0}$}&\gate{H} & \gate{Z} & \gate{H} & \gate{X} & \gate{Z} & \gate{X} & \gate{H}& \meter{}
\end{quantikz}

    \caption{Circuit 1.}
        \label{fig:circuit_1}
\end{figure}
We are given the circuit in \cref{fig:circuit_1}. In the following we see the state of 
the circuit in each slice.
We start from the state
\begin{equation}
    \ket{\psi_0} = \ket{00}  \, .
\end{equation}
Then:
\begin{equation}
        \ket{\psi_1} = \ket{++} = \frac{1}{2} \left( \ket{00} + \ket{01} + \ket{10} + \ket{11} \right) = \sum_{n=0}^{3} \ket{n} \, ,
\end{equation}
\begin{equation}
        \ket{\psi_2} = \frac{1}{\sqrt{2}} \left( \ket{0+} + \ket{1-}\right) \, ,
\end{equation}
\begin{equation}
        \ket{\psi_3} = \frac{1}{\sqrt{2}} \left( \ket{+0} + \ket{-1} \right) \, ,
\end{equation}
\begin{equation}
        \ket{\psi_4} = \frac{1}{\sqrt{2}} \left( \ket{+1} - \ket{-0} \right)  \, ,
\end{equation}
Since CZ is symmetric:
\begin{equation}
        \ket{\psi_5} =  \frac{1}{\sqrt{2}} \left( \ket{-1} - \ket{-0} \right) =  - \ket{--} \, ,
\end{equation}
\begin{equation}
        \ket{\psi_6} = - \ket{--} \, ,
\end{equation}
\begin{equation}
        \ket{\psi_7} = - \ket{11} \, .
\end{equation}

