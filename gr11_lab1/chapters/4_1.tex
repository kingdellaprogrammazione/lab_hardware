\begin{figure}[h]
    \centering
\begin{quantikz}
\lstick{$\ket{0}$}   & \gate{H} \slice{$\ket{\psi_{t1}}$} & \ctrl{1} \slice{$\ket{\psi_{t2}}$} & \gate{H} \slice{$\ket{\psi_{t3}}$} & \gate{X}  \slice{$\ket{\psi_{t4}}$} & \qw & \ctrl{1} & \qw \slice{$\ket{\psi_{t5}}$} & \gate{X} \slice{$\ket{\psi_{t6}}$} & \gate{H} \slice{$\ket{\psi_{t7}}$} &  \meter{} \\
\lstick{$\ket{0}$}&\gate{H} & \gate{Z} & \gate{H} & \gate{X} &  \gate{H} & \targ{}  & \gate{H}  & \gate{X} & \gate{H}& \meter{}
\end{quantikz}

\end{figure}


\subsection{Evaluation by hand of the state evolution of a circuit}
Since we have that
\begin{equation}
    (I \otimes H) (CX_{1 \to 2}) ( I \otimes H) = CZ_{1 \to 2}
\end{equation}
The circuit is equivalent to the one we analyzed in section ... 

The demonstration is as follow: we check the action on a basis set: 
\begin{equation}
    \ket{0+}, \ket{0-}, \ket{1+}, \ket{1-}.
\end{equation}
The states with the control qubit set to $\ket{0}$ are unaltered, since $HH=I$, like in the $CZ_{1\to2}$. 
When the control qubit is set to $\ket{1}$, we have:
\begin{align}
    \ket{1+} \to \ket{1-} \\
    \ket{1-}  \to \ket{1+}.
\end{align}
Summarizing, te full action is:
\begin{align}
    \ket{0+} \to \ket{0+} \\
    \ket{0-}  \to \ket{0-} \\
    \ket{1+} \to \ket{1-} \\
    \ket{1-}  \to \ket{1+}.
\end{align}
And this is the exact action of the $CZ_{1 \to 2}$ gate in the circuit before.

