\begin{lstlisting}[language=Python, caption={Qiskit code}, label={lst:qiskit-circuit}]
import qiskit
from qiskit import QuantumCircuit, transpile
from qiskit_aer import AerSimulator
from qiskit.visualization import plot_histogram
import matplotlib.pyplot as plt
from qiskit.qasm3 import dumps

# step 0: create a Quantum Circuit
qc = QuantumCircuit(2, 2)  # 2 qubit, 2 bit classici
# step 1:
qc.h(0)
qc.h(1)
qc.barrier(0, 1)
#step 2:
qc.cz(1, 0)

qc.barrier(0, 1)
# step 3:
qc.h(0)
qc.h(1)

qc.barrier(0, 1)
#stp 4:
qc.x(0)
qc.x(1)

qc.barrier(0, 1)
# step 5:
qc.h(1)

qc.barrier(0, 1)
# step 6:
qc.cx(0, 1)

qc.barrier(0, 1)
# step 7:
qc.h(1)

qc.barrier(0, 1)
# step 8:
qc.x(0)
qc.x(1)

qc.barrier(0, 1)
#step 9:
qc.h(0)
qc.h(1)

# measure q[0] -> c[0]; measure q[1] -> c[1];
qc.measure(0, 0)
qc.measure(1, 1)

qc.draw('mpl')        # diagramma grafico
\end{lstlisting}



\begin{figure}[H]
    \centering
    \includegraphics[width=1\textwidth]{images/circuit_es4.png}
    \caption{Qiskit circuit}
    \label{fig:circuit es4}
\end{figure}




%% Qasm 
\begin{lstlisting}[language=Python, caption={Export circuit to QASM3}, label={lst:qasm-export}]
qasm_str = dumps(qc)
print(qasm_str)
\end{lstlisting}

\begin{lstlisting}[language=Python, caption={Output generato: OpenQASM 3}, label={lst:qasm3-output}, backgroundcolor=\color{bg}]
OPENQASM 3.0;
include "stdgates.inc";
bit[2] c;
qubit[2] q;
h q[0];
h q[1];
barrier q[0], q[1];
cz q[1], q[0];
barrier q[0], q[1];
h q[0];
h q[1];
barrier q[0], q[1];
x q[0];
x q[1];
barrier q[0], q[1];
h q[1];
barrier q[0], q[1];
cx q[0], q[1];
barrier q[0], q[1];
h q[1];
barrier q[0], q[1];
x q[0];
x q[1];
barrier q[0], q[1];
h q[0];
h q[1];
c[0] = measure q[0];
c[1] = measure q[1];
\end{lstlisting}